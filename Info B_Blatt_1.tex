
  \documentclass[a4paper, 10pt, landscape]{scrartcl}
  \usepackage[utf8]{inputenc}
  \usepackage[ngerman]{babel}
  \usepackage{amsfonts}
  \usepackage{amssymb}
  \usepackage{amsmath}
  \usepackage{geometry}
  \usepackage{longtable}
  \usepackage{colortbl}
  \usepackage{xcolor}

  \geometry{a4paper,left=1cm,right=1cm, top=1cm, bottom=1cm}
  \pagenumbering{gobble}

  \begin{document}
\ttfamily \huge{Info B} Testat Blatt 1
      \small 
\begin{longtable}{|p{6cm}|c|c|c|c|c|c|c|c|c|c|c|c|c|c|p{8cm}|}\hline
& \textbf{P} & \textbf{G1} & \textbf{G2} & \textbf{G3} & \textbf{G4} & \textbf{G5} & \textbf{G6} & \textbf{G7} & \textbf{G8} & \textbf{G9} & \textbf{G10} & \textbf{G11} & \textbf{G12} & \textbf{G13}& Comments \\ \hline\hline
\rowcolor{gray!50}\textbf{Aufgabe 1:Sichtbarkeit} & 20& & & & & & & & & & & & & & \\\hline\hline
01-05& 5& & & & & & & & & & & & & & \\ \hline
06-10& 5& & & & & & & & & & & & & & \\ \hline
11-15& 5& & & & & & & & & & & & & & \\ \hline
Call-by-value vs Call-by-reference& 5& & & & & & & & & & & & & & \\ \hline
\rowcolor{gray!50}\textbf{Aufgabe 2:Fibonacci} & 30& & & & & & & & & & & & & & \\\hline\hline
Datenfelder nMinus1 + nMinus2& 2& & & & & & & & & & & & & & \\ \hline
beide private& 2& & & & & & & & & & & & & & \\ \hline
im Konstruktor initialisiert& 2& & & & & & & & & & & & & & \\ \hline
Methode next, fn = fn-1 + fn-2, nMinus1/2 neu& 4& & & & & & & & & & & & & & \\ \hline
FibonacciPrint mit Ausgabe bei Fehler& 3& & & & & & & & & & & & & & \\ \hline
Anleitung/Help& 2& & & & & & & & & & & & & & \\ \hline
einlesen Kommandozeilenparam (parseInt)& 3& & & & & & & & & & & & & & \\ \hline
Formatierung Output& 4& & & & & & & & & & & & & & \\ \hline
mit printf& 4& & & & & & & & & & & & & & \\ \hline
JavaDoc, sprechende Varnamen etc.& 4& & & & & & & & & & & & & & \\ \hline
\rowcolor{gray!50}\textbf{Aufgabe 3:Fraction} & 30& & & & & & & & & & & & & & \\\hline\hline
Numer + Denom private& 1& & & & & & & & & & & & & & \\ \hline
Konstruktor ein Arg& 1& & & & & & & & & & & & & & \\ \hline
Verkettung mit anderem Konstruktor& 2& & & & & & & & & & & & & & \\ \hline
Konstruktor zwei Args& 2& & & & & & & & & & & & & & \\ \hline
Prüfen Nenner 0& 1& & & & & & & & & & & & & & \\ \hline
ggT erzeugen& 1& & & & & & & & & & & & & & \\ \hline
VZ-Test (auch implizit durch ggT)& 2& & & & & & & & & & & & & & \\ \hline
Zuweisung, teilen durch GG & 2& & & & & & & & & & & & & & \\ \hline
Multipl mit int, fraction, vielen frcts& 3& & & & & & & & & & & & & & \\ \hline
Dividieren durch andere Fraction& 1& & & & & & & & & & & & & & \\ \hline
Jeweils neue Fraction& 2& & & & & & & & & & & & & & \\ \hline
toString& 1& & & & & & & & & & & & & & \\ \hline
FractionTest alles durchprobiert (auch kürzen + VZ)& 4& & & & & & & & & & & & & & \\ \hline
	extit{automatische Tests}& 4& & & & & & & & & & & & & & \\ \hline
JavaDoc, sprechende Varnamen etc.& 4& & & & & & & & & & & & & & \\ \hline
\rowcolor{gray!50}\textbf{Aufgabe 4:Fragen} & 20& & & & & & & & & & & & & & \\\hline\hline
OO != OOP& 4& & & & & & & & & & & & & & \\ \hline
Klasse, Objekt/Instanz& 4& & & & & & & & & & & & & & \\ \hline
Klassenvariable, Instanzvariable& 2& & & & & & & & & & & & & & \\ \hline
Klassenmethode, Instanzmethode& 2& & & & & & & & & & & & & & \\ \hline
Identität von einem Objekt (this + this.)& 2& & & & & & & & & & & & & & \\ \hline
Methodenkopf + Rumpf& 2& & & & & & & & & & & & & & \\ \hline
Methodensignatur& 2& & & & & & & & & & & & & & \\ \hline
Formaler vs aktueller Parameter& 2& & & & & & & & & & & & & & \\ \hline

\rowcolor{gray!10}\textbf{Summe} & 100 & & & & & & & & & & & & & & \\ \hline
\end{longtable}
\begin{longtable}{l l }
\textbf{G1: }Di, 11:00: Freya Berstermann, Irina Vortkamp\\\textbf{G2: }Di, 11:30: Felix Nardmann, Benjamin Tolksdorf\\
\textbf{G3: }Di, 14:00: Daniel Stattkus, Rene Brinkhege\\\textbf{G4: }Di, 14:30: Daniel Pohl, Bianca Krömer\\
\textbf{G5: }Di, 16:00: Darren Kopatz, Alexander Miller\\\textbf{G6: }Di, 11:00: Freya Berstermann, Irina Vortkamp\\
\textbf{G7: }Di, 11:30: Felix Nardmann, Benjamin Tolksdorf\\\textbf{G8: }Di, 14:00: Daniel Stattkus, Rene Brinkhege\\
\textbf{G9: }Di, 14:30: Daniel Pohl, Bianca Krömer\\\textbf{G10: }Di, 16:00: Darren Kopatz, Alexander Miller\\
\textbf{G11: }Di, 14:00: Daniel Stattkus, Rene Brinkhege\\\textbf{G12: }Di, 14:30: Daniel Pohl, Bianca Krömer\\
\textbf{G13: }Di, 16:00: Darren Kopatz, Alexander Miller\\
\end{longtable}
  \end{document}